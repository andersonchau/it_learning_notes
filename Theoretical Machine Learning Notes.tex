\documentclass{article}
\usepackage{amsmath}
\usepackage{blindtext}
\begin{document}

\section{Foreword}
Author : Anderson Chau
\newline 
\newline 
Disclaimer : This notes is written only for my own memorization purpose after I have studied online lecture notes and blogs. 
% ######################################################################
% ######################################################################
% ######################################################################
\section{K-Means Clustering}
Simple Description : Identify clusters by finding the centroid of data points
\newline
\newline
Algorithm :  
\newline
\newline
1. Initialize \( \mu_1 \),\( \mu_2 \).... \( \mu_k \) randomly (k is hyper-parameter) 
\newline
\newline
2. Repeated until converge : 
\newline
\newline
(i) \( $$c^{(i)} := \arg \min_j ||x^{(i)} - \mu_j||^2$$ , j \in [1:k]\) , (i.e. \(c^{(i)}\) denote which  \( \mu \)  the \(x^{(i)}\) is linked to. Link each data point to nearest  \( \mu_j \). If \( x^{(i)}\) is nearest to \( \mu_s \), then \(c^{(i)} = j\). Thus, k partitions are created.  )
\newline
\newline
(ii) \($$\mu_j := \frac{\sum_{i=1}^m 1\{{c^{(i)} = j}\} x^{(i)}}{\sum_{i=1}^m 1\{{c^{(i)} = j}\}}$$\) , (i.e. For each data point in each partition from (i) , find the new centroid and assign to \(\mu_k\)
\newline
\newline 
Proof of convergence of the algorithm : consider 
\newline 
\newline 
\( $$J(c, \mu) = \sum_{i=1}^m ||x^{(i)} - \mu_{c^{(i)}}||^2$$\)
\newline 
Observation : J must be monotonically decreasing. It is because for step (i) It is adjusting \(c^(i)\) to reduce J, for step (ii) we are adjusting  \(\mu_j\) to reduce J
% ######################################################################
% ######################################################################
% ######################################################################
\section{Linear Regression(MSE approach) }
Hypothesis : 
\[
h_\theta(x) = \sum_j \theta_j x_j = \theta^\top x
\]
We want to minimize MSE (Mean Square Error) 
\[
J(\theta) = \frac{1}{2} \sum_i \left( h_\theta(x^{(i)}) - y^{(i)} \right)^2 = \frac{1}{2} \sum_i \left( \theta^\top x^{(i)} - y^{(i)} \right)^2
\]
Gradient of J : 
\[
\frac{\partial J(\theta)}{\partial \theta_j} = \sum_i x_j^{(i)} \left( h_\theta(x^{(i)}) - y^{(i)} \right)
\]
Each \(\theta_j\) is updated for each step by gradient descent algorithm. 
\[
\theta_j := \theta_j - \alpha \frac{\partial}{\partial \theta_j} J(\theta)
\]
Practically : 
\newline
(i) Learning rate \(\alpha\) is hyperparameter and data dependent , larger, fewer steps to get to min. but may miss the minimum. (Monitor the loss curve, J value vs iteration ). 
\newline
(ii) Batch GD is slow, may be Mini-Batch GD or Stochastic GD.
\newline
(iii) If \(\alpha\) is small but the loss oscillate , converged and stop learning.


\section{Linear Regression(MLE approach) }
\section{Logistic Regression}
\[P(y=1|x) = h_\theta(x) = \frac{1}{1 + \exp(-\theta^\top x)} \equiv \sigma(\theta^\top x)\]
\[P(y=0|x) = 1 - P(y=1|x) = 1 - h_\theta(x)\]
Loss function is 
\[J(\theta) = -\sum_i \left[ y^{(i)} \log(h_\theta(x^{(i)})) + (1 - y^{(i)}) \log(1 - h_\theta(x^{(i)})) \right]\]
Again , BGD for following gradient of J : 
\[\frac{\partial J(\theta)}{\partial \theta_j} = \sum_i x^{(i)}j \left( h\theta(x^{(i)}) - y^{(i)} \right)\]
Interpertation : For a particular sample : if h return 1/0 and y return 1/0 , the term is 0. if  h return 1/0 and y return 0/1 , the term is positive infinity. 
\section{Logistic Regression(MLE approach)}
\section{Softmax Regression(Multi-Class Logistic)}
k classes, n x k parameters , and the hypothesis is :

\begin{align}
h_\theta(x) =
\begin{bmatrix}
P(y = 1 | x; \theta) \\
P(y = 2 | x; \theta) \\
\vdots \\
P(y = K | x; \theta)
\end{bmatrix}
=
\frac{1}{ \sum_{j=1}^{K}{\exp(\theta^{(j)\top} x) }}
\begin{bmatrix}
\exp(\theta^{(1)\top} x ) \\
\exp(\theta^{(2)\top} x ) \\
\vdots \\
\exp(\theta^{(K)\top} x ) \\
\end{bmatrix}
\end{align}

Below Loss function  is simple to understand (compare with previous hypothesis)  : we want to maximize the y=k associated probility. 
\[J(\theta) = -\left[ \sum_{i=1}^m \sum_{k=1}^K 1\{y^{(i)}=k\} \log \left( \frac{\exp(\theta^{(k)\top} x^{(i)})}{\sum_{j=1}^K \exp(\theta^{(j)\top} x^{(i)})} \right) \right]\]
Gradient of J is, we solve the problem by  GD : 
\[\nabla_{\theta^{(k)}} J(\theta) = -\sum_{i=1}^m \left[ x^{(i)} \left( 1\{y^{(i)}=k\} - P(y^{(i)}=k|x^{(i)};\theta) \right) \right]\]
Where : 
\[P(y^{(i)}=k|x^{(i)};\theta) = \frac{\exp(\theta^{(k)\top} x^{(i)})}{\sum_{j=1}^K \exp(\theta^{(j)\top} x^{(i)})}\]

% ######################################################################
% ######################################################################
% ######################################################################
\section{Loss function in Classification(Binary) Problem - General treatment}
General Hypothesis : \(h_\theta(x) = x^T \theta\)
\newline
\newline
Adjustment for binary classification : 
\newline
\(
\text{sign}(h_\theta(x)) = \text{sign}(\theta^T x)= \text{sign}(t) = 
\begin{cases} 
1 & \text{if } t > 0 \\
0 & \text{if } t = 0 \\
-1 & \text{if } t < 0 
\end{cases}
\)
\newline
\newline
Measure of confidence :  \(h_\theta(x) = x^T \theta\) gives larger value, more confident  
\newline
\newline
Margin ( \(y x^T \theta\) ) : (i) if \(h_\theta(x)\) classify correctly, margin is positive, otherwise negative.
\newline
(ii) Therefore our objective is to maximize the margin ( we want both correct classification and be confident)
\newline
\newline
Consider the following loss function : 
\[
J(\theta) = \frac{1}{m} \sum_{i=1}^{m} \phi\left(y^{(i)} \theta^T x^{(i)}\right)
\]

We want penalize wrong classfication and encourage correct one , we design \(\phi\) as 
\(
\phi(z) \to 0 \text{ as } z \to \infty, \quad \text{while} \quad \phi(z) \to \infty \text{ as } z \to -\infty
\)  where  \(z = yx^{T}\theta\) , and examples are :  
\newline 
\newline
logistic loss : \(\phi_{\text{logistic}}(z) = \log(1 + e^{-z})\) ,used in logistic regression 
\newline 
\newline 
hinge loss : \(\phi_{\text{hinge}}(z) = [1 - z]_+ = \max{1 - z, 0}\), used in SVM
\newline 
\newline 
Exponential loss \(\phi_{\text{exp}}(z) = e^{-z}\), used in boosting 
\section{Kernel}
(I) To x map from lower higher dimension. Useful when data are non-linearly separable(Transform to a curve) 
\newline
(II) Computation complexity not necessarily increase proportionately.
\newline
(III) Example : a mapping function \(\varphi : R \to R^{4}\) , \(x \to [1,x,x^{2},x^{3}]\), and h is \(\theta^{T}x\) having  \(\theta  = [\theta_1,\theta_2,\theta_3,\theta_4]\) 
\newline
(IV) x is called attribute, \(x \to [1,x,x^{2},x^{3}]\) called feature, \(\varphi\) feature map
\newline
(IV) Another Example : a mapping function \(\varphi : R \to R^{1000}\) , \(x \to [1,x_1,x_1^{2},x_1^{3},x_1 x_2,x_1 x_2^{2} .... ]\)

 
\end{document}
